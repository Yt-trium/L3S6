\documentclass[12pt]{report}
\usepackage[utf8]{inputenc}
\usepackage[T1]{fontenc}
\usepackage[french]{babel}
\usepackage{array}
\usepackage{color}
\usepackage{hyperref}
\usepackage{amssymb}
\usepackage{tikz}
\usepackage{stmaryrd}

\usetikzlibrary{arrows}

\title{Théorie des langages}
\date{S6 Printemps 2017}
\author{MEYER Cyril\\D'après le cours de TAJINE Mohamed}

\begin{document}
\selectlanguage{french}

\maketitle
\newpage
\tableofcontents
\newpage

\chapter{Ensemble, Relation, Fonction et Langages}

\section{Définition formel de la notion de langage}
	\begin{itemize}
	\item Ensemble de mot sur un alphabet donné
	\item Base de toutes les activités humaines
	\\ Intérêt dans notre cas
	\item Reconnaissance d'un mot d'un langage
	\item Génération d'un mot
	\item Classification des langages
\end{itemize}

\center
\begin{tikzpicture}[auto]
    \node[draw] at (0, 0) (a) {Algorithme};
    \node[draw] at (5, 0) (p) {Programme};
    \path[->] (a) edge node {Langage} (p);
\end{tikzpicture}

\flushleft
Exemple sur le 10e problème d'Hilbert :
\paragraph{}
On donne une équation de Diophante à un nombre quelconque d'inconnues et à coefficients entiers rationnels : On demande de trouver une méthode par laquelle, au moyen d'un nombre fini d'opérations, on pourra distinguer si l'équation peut être résolue en nombre entiers rationnels.
\paragraph{}
C'est à dire trouver un algorithme pour dire si oui ou non ce genre de polynôme peut être résolue en nombre entiers rationnels :\\
\paragraph{}
\begin{math}
P(a_1,a_2,...,a_n,x_1,x_2,...,x_m) = 0
\end{math}
où P est un polynôme a plusieurs variables et à coefficients entiers.
Les $a_i$ sont les les inconnus, les $x_j$ les paramètres.

\paragraph{}
Pour résoudre ce problème, Church, Turing et d'autres ont du définir formellement ce qu'est un algorithme.
On parlait à l'époque de fonction primitives récursives, $\lambda$-calcul mais on parlera dans ce cours d'algorithmes.

\paragraph{}
Définition précise d'un langage de programmation
-> Outils pour vérifier la syntaxe d'un programme écrit dans un langage de programmation.

\begin{itemize}
	\item Compilateur
	\item Analyse syntaxique / lexicale
\end{itemize}
\paragraph{}
\begin{tabular}{|l|l|l|}
	\hline
	Classe de langage & Reconnaissance de mot & Génération de mot \\
	\hline
	Langage régulier (rationnel) & Automate fini & Grammaire régulière \\
	\hline
	Langage algébrique & Automate à Pile & Grammaire algébrique \\
	\hline
	Langage récursif & Machine de Turing & Grammaire générale \\
	\hline
\end{tabular}

\center
\begin{tikzpicture}
\node[draw,fill=blue!30,text depth=10cm,minimum width=10cm,font=\Large](main){Tous les langages};
\node[draw,fill=green!30,text depth=7cm,minimum width=7cm,font=\Large] at (main.center){Récursif};
\node[draw,fill=red!30,text depth=5cm,minimum width=5cm,font=\Large] at (main.center){Algébrique};
\node[draw,fill=white!30,text depth=2.5cm,minimum width=2.5cm,font=\Large] at (main.center){Régulier};
\end{tikzpicture}
\flushleft
\section{Ensemble}
	\paragraph{}
Ensemble : un ensemble est une "collection" d'objet.\\
Exemples : Les entiers 1,3,5,7,9 constitue une collection des impaires $\leq$ 9, on peut le noter $\{1,3,5,7,9\}$\\
L'ordre des éléments n'est pas important.\\
Il ne peut pas y avoir de répétition.\\
Exemple d'ensembles connu : $\mathbb{N}, \mathbb{Z}, \mathbb{Q}, \mathbb{R}, \mathbb{C}$ 

\subsection{Paradoxe de Russel}
\begin{math}
	R = \{x | x \notin x\},
	R \in R => R \notin R,
	R \notin R => R \in R
\end{math}
Le paradoxe a conduit a une avancé dans les axiom de la théorie des essembles (Zermelo Fraenkel)

\paragraph{}
\subsection{Opération sur les ensembles}
Pour dire qu'un élément $e$ est dans un ensemble $E$, on écrit $e \in E$\\
Pour dire qu'un ensemble $E$ est inclus dans un ensemble F on écrit $E \subseteq F$\\
Lorsque $E \neq F$ et $E \subseteq F$ on écrit $E \subset F$\\
\paragraph{}
Soit $A$ et $B$ deux ensemble.
\begin{itemize}
	\item non$(e \in B) = e \notin B$
	\item $A \cup B = \{e | e \in A$ ou $e \in B\}$ = Union
	\item $A \cap B = \{e | e \in A$ et $e \in B\}$ = Intersection
	\item $A \setminus B = \{e | e \in A$ et non$(e \in B)\}$
	\item $P(A) = \{E | E \subseteq A\}$ = l'ensemble des parties de A.
	\item $A \times B = \{(a,b) | a \in A $et$b \in B\}$ le produit cartésien de A,B
	\\(couple $(a,b) -> \{a,\{a,b\}\}$)
	\item $A_1, A_2, ..., A_n , n\geq$ sont des ensembles, alors\\
	$A_1 \times A_2 \times ... \times A_n = \{(a_1, a_2, ..., a_n) | a_1 \in A_1 \times ... \times a_n \in A_n\}$
\end{itemize}

\paragraph{}
Définition : soient A un ensemble et $\Pi \subseteq P(A)$\\
$\Pi = \{B_1, ..., B_n, ...\}$ est une portion de A si $B_i \cap B_j = \emptyset$\\
$B_i \neq \emptyset$ pour tout $i$\\
$A = B_1 \cup B_2 \cup ... \cup B_n \cup ... = U_{iEI}B_i$\\
$A = \{1,3,5,7,9\}$
Si $\Pi = \{\{1,5\},\{7\},\{3,9\}\}$, alors $\Pi$ est une partition de $A$.\\

\paragraph{}
Remarque : soient $A$, $B$ et $C$ des ensembles.
\begin{enumerate}
	\item $A = B <=> A \subseteq B $ et $ B \subseteq A$
	\item $A \setminus (B \cup C) = (A \setminus B) \cap (A \setminus C)$
	\item $A \setminus (B \cap C) = (A \setminus B) \cup (A \setminus C)$
	-> Ce sont les lois de De Morgan.
\end{enumerate}
\section{Relation et Fonction}
	
Soient $A_1,A_2, ...,A_n\;n$ ensembles $(n\geq1)$
Une relation n-aire (d'arité n) $R$ est un sous ensemble de $A_1 \times A_2 \times ... \times A_n\;(R\subseteq A_1 \times A_2 \times ... \times A_n)$
Pour $n=2 R$ est dite une relation binaire.\\
\paragraph{}
Exemple :\\
$R_1 = \{a \in \mathbb{Z} \;|\;a $ est impaire $\} \subseteq \mathbb{Z}$ (relation d'arité 1) $= \{...,-5,-3,-1,1,3,5,...\}$\\
$R_2 = \{(a,b)\in\mathbb{Z}^2 \;|\; a \leq b\}$ (relation d'arité 2) sur $\mathbb{Z}^2$\\
$R_3 = \{(a,b,c)\in\mathbb{Z}^3 \;|\; a \leq b \leq c \}$ (relation d'arité 3) sur $\mathbb{Z}^3$\\

\paragraph{}
Définition (relation fonctionnelle)
Soit $R$ une relation n-aire sur $A_1,A_2, ...,A_n\;n$ pour $n \geq 2$\\
Si pour tout (n-1)-uplet $(a_1,...,a_{n-1})\in A_1 \times A_2 \times ... \times A_{n-1}$ il existe un et un seul élément $a_n \in A_n$ tel que $(a_1,...,a_{n-1},a_n) \in R$ on dit alors que $R$ est une relation fonctionnelle relativement à $A_1 \times ... \times A_{n-1}$ et on utilise la notation $f_R : A_1 \times .. \times A_{n-1} \to A_n$ et pour tout $(a_1, ..., a_{n-1}) \in A_1 \times ... \times A_{n-1}$ l'unique élément $a_n \in A_n$ tel que $(a_1,...,a_{n-1},a_n) \in R sera noté f_R (a_1,..., a_{n-1})$\\
$(a_n = f_R (a_1,...,a_{n-1}))$\\
$f_R$ est appelé la fonction associée à la relation fonctionnel R.\\
$A_1 \times ... \times A_{n-1}$ est appelé le domaine de $f_R$ (ensemble de départ)\\
$A_n$ est appelé l'ensemble d'arriver de $f_R$.\\
\paragraph{}
Exemple de relation fonctionnelle.

\begin{itemize}
	\item $R_1 = \{(x,y)\;|\; x \in A\}$ ($f_{R_1}$ est appelé identité sur $A$)
	\item $R_2 = \{(x,y,x+y)\;|\; x,y \in \mathbb{Z}\}$\\
	($f_{R_2} : \mathbb{Z} \times \mathbb{Z} \to \mathbb{Z}$ avec $f_{R_2}(x,y)=x+y$)
\end{itemize}

\paragraph{}
Remarque : chaque fonction correspond à une et une seul relation fonctionnel.\\
a $f: A_1 \times ... \times A_{n-1} \to A_n$ on associe la relation :\\
$R_f = \{(a_1,...,a_{n-1}, f(a1,...,a_{n-1}))\;|\; (a_1,...,a_{n-1}) \in A_1 \times ... \times A_{n-1}\}$

\paragraph{}
Définitions : soit $A$ et $B$ deux ensembles et $f : A \to B$

\begin{enumerate}
	\item f est dite injective si $\forall a, a' \in A$ si $a \neq a'$ alors $f(a) \neq f(a')\;
	(\to\; $on dit que f est une injection de A dans B)
	\item f est dite surjective si $\forall b \in A, \exists a \in A $ tel que $f(a) = b\;
	(\to\; $on dit que f est une surjection de A dans B$)$
	\item f est dite bijective si elle est à la foi injective et surjective.
	$(\to\; $on dit que f est une bijection de A dans B$)$
\end{enumerate}

Lorsque $f$ est bijective alors $f^{-1}$ est la fonction de $B \to A$ tel que $\forall a \in A, \forall b \in B$, $f(a) = b$ si et seulement si $f^{-1}(b) = a$

\section{Cardinalité}
	\input{cardinalite}
\paragraph{}
Rappel de notation :\\
Soit $a$ et $b \in \mathbb{Z}\;\llbracket a,b \rrbracket$ = $\{n \in \mathbb{Z}\;|\; 0\leq n\leq b\}$\\
$\to \llbracket \rrbracket$ intervale entier\\
$\to \llbracket 1,0 \rrbracket = \{\} = \emptyset$\\
\paragraph{}
Définition : deux ensemble $A$ et $B$ sont dit de même cardinalité si il existe une bijection de f de A dans B.
\paragraph{}
Remarque : être de même cardinalité est une relation symétrique.\\
Un ensemble A est dit de cardinal fini si il existe un entier $n \geq 0$ tel qu'il existe une bijection : $\llbracket 1,n \rrbracket \to A$ et donc ce $n$ est appelé le cardinal de $A$
\paragraph{}
Notes : Le cardinal = le nombre d'élément de l'ensemble fini\\
\paragraph{}
Cardinal non fini :\\
Un ensemble A est dit de cardinal infini si A n'est pas de cardinal fini.\\
Un ensemble est dit infini dénombrable s'il existe une bijection $f\; N \to A$\\

\paragraph{}
Exemple :\\
$P = \{2n\;|\;n \in \mathbb{N}\} \subset \mathbb{N}$\\
$f : \mathbb{N}\to P\;\;\;n\to 2n$\\
$f$ est bijective et donc $P$ est un ensemble infini dénombrable.\\$\mathbb{N} \times \mathbb{N} = \mathbb{N}^2$ est un ensemble infini dénombrable.\\
$f(i,j)=1+2+...+(i+j))+i= (\sum_{k=1}^{i+j}k)+i $\\$= 1/2 (i+j)(i+j+1)+i $\\$=1/2 ((i+j)^2+3i+j)$\\
Montrons alors que $f$ est une bijection.

$\to f$ surjectif
Soit $h \in \mathbb{N}$, supposons qu'il existe $(i,j) \in \mathbb{N}^2$ tel que $f(i,j) = k$\\
$\exists!$ (existe-t-il) $(i',j') \in \mathbb{N}^2$ tel que $f(i',j')=k+1$\\
Deux cas possible :\\

1. $j>0$ alors $f(i+1,j-1)=(\sum_{h=0}^{(i+j)+(j-1)}h)+i+1$\\
$=((\sum_{h=1}^{i+j}h)+i)+1 = k+1 = f(i,j)+1$\\
2. $j=0$ alors $f(j,i+1)=(\sum_{h=1}^{(j+i+1)}h)+j$\\
$=(\sum_{h=1}^{i+1}h) = (\sum_{h=1}^{i}h)+(i+1)$\\
$=((\sum_{h=1}^{i+j}h)+i)+1$\\
$=f(i,j)+1 = k+1$\\
\paragraph{}
Par induction, comme f(0,0) = 0 et on peut trouver k+1, f est surjectif.

$\to f$ injectif

\end{document}